\chapter*{Conclusioni}
\noindent L'obiettivo della tesi è l'illustrazione di un approccio avanzato per l'analisi delle interazioni molecolari nell'ambito del docking computazionale. L'approccio avanzato proviene dal Deep Learning, una branca dell'Intelligenza Artificiale, nella quale le reti neurali convoluzionali, dopo una fase di apprendimento in cui estraggono da sé le caratteristiche rilevanti della conformazione proteina-ligando, effettuano una predizione sulla miglior posa del ligando secondo particolari metriche, tenendo conto di alcune limitazioni come la rigidità delle strutture, ed effettuano valutazioni energetiche a partire da tale conformazione. 
Le valutazione delle prestazioni della rete neurale convoluzionale applicata al docking computazionale non è abbastanza efficace se non è introdotto un termine di paragone, che riesca a fornire profondità allo studio di questo approccio avanzato. A tal proposito, è stato selezionato un tradizionale software di docking computazionale, basato su un approccio empirico, spesso utilizzato in analisi di questo genere.
A questo punto, per poter valutare ed analizzare le prestazioni, sono state individuate delle procedure di preparazione del dataset iniziale e delle procedure di analisi delle interazioni prodotte, generalmente comuni al processo di analisi molecolare computazionale ma applicate al caso specifico, in maniera tale da osservare qualitativamente e quantitativamente le interazioni molecolari interessanti. 
I risultati sperimentali dicono che le prestazioni della rete neurale convoluzionale applicata all'interno di GNINA, configurato in maniera predefinita, danno luogo a risultati estremamente diversi rispetto a quelli forniti da un software tradizionale per analisi di blind docking. 
Le predizioni effettuate da GNINA risultano essere migliori in termini di RMSD ma peggiori in termini di affinità. Dal punto di vista delle interazioni molecolari nel complesso però è stato dimostrato che, sebbene siano quantitativamente inferiori, queste siano qualitativamente migliori, come testimonia la parità rispetto all'Hydrogen Bonds Ratio.
Ciò significa che una bassa RMSD non implica necessariamente un'affinità migliore e che un'affinità migliore non implica necessariamente legami più stabili ed efficaci. 
Nonostante il numero di legami osservati sia inferiore, è emerso che il numero di residui coinvolti in legami è notevolmente maggiore, producendo quindi heatmaps più ampie e sparse in cui le interazioni sono maggiormente distribuite, rispetto a quelle ridotte e fitte di AutoDock Vina, in cui è maggiormente possibile approssimare l'insieme dei residui coinvolti in interazioni ad un sotto-insieme ridotto di residui interessanti. 
Calando queste considerazioni nel contesto biochimico, è necessario però tenere conto del tipo di applicazione che si vuole effettuare e delle limitazioni che comporta un approccio di questo tipo. La mole di dipendenze, la necessità di una notevole potenza di calcolo e la quantità di tempo richiesta per ultimare l'intero processo di esperimenti di docking sono degli aspetti da riguardare fortemente rispetto ai relativi competitors, sebbene i risultati sperimentali mostrino una predizione più accurata. 
La soluzione potrebbe essere l'utilizzo complementare dei due approcci per sfruttare le potenzialità di entrambi: s'intende una prima fase di \textit{fast screening} utilizzando, ad esempio AutoDock Vina, ottimo nelle valutazioni di blind docking, sulla base della quale effettuare l'estrazione dei residui caratterizzanti delle interazioni osservate e, successivamente eseguire una seconda fase di \textit{redocking} nei siti di tali residui per ottenere predizioni più affidabili ed accurate, attraverso l'utilizzo della rete neurale convoluzionale, come nel caso di GNINA.
\section*{Sviluppi futuri}
\noindent L'elaborato proposto tenta di rispondere ai quesiti relativi all'applicazione delle reti neurali convoluzionali, limitandosi all'utilizzo di una configurazione elementare e delineandone punti di forza e debolezze, sebbene ponga diversi spunti per effettuare ulteriori analisi, a partire dalle considerazioni emerse. In prima istanza, per completare l'analisi proposta dall'elaborato, sarebbe interessante seguire la soluzione ideale sopracitata, effettuando un docking più specifico nei siti dei residui maggiormente coinvolti, applicando la rete neurale convoluzionale ed osservandone poi le differenze in termini di risultati.

\noindent Al fine di testare il modello in relazione all'affinità predetta, sarebbe curioso riprodurre l'esperimento applicando diverse migliorie, per esempio, configurandolo in maniera tale da minimizzare il parametro Energy, corrispondente all'affinità, piuttosto che il parametro CNNscore, come proposto. Alternativamente, è possibile riprodurre il medesimo esperimento attraverso un modello come HiRes Affinity, menzionato nella Figura \ref{fig:models}

\noindent Una soluzione più intrigante e complessa ma specifica dell'applicazione che ne si vuole fare consiste nell'addestramento dei modelli built-in implementati in GNINA, partizionando il dataset iniziale secondo i classici approcci statistici insiti nel Machine Learning e nell'apprendimento supervisionato. 

\noindent L'elaborato, inoltre, tiene in considerazione un unico termine di paragone per valutare le prestazioni del modello. Nulla vieta la comparazione rispetto ad altre funzioni di scoring e/o altri software di docking di comune utilizzo.

\noindent Come sottolineato a più riprese, l'applicazione proposta è parte di un lavoro iniziato nel marzo 2021 attraverso l'attività di Tirocinio, svolta assieme al collega Alfredo Mungari e sotto la supervisione del Prof. Angelo Ciaramella, del Dott. Ferdinando Febbraio e della Dott.ssa Mónica del Águila, nello sviluppo di un'applicazione di supporto in ambito bioinformatico.

\noindent I dettagli sono disponibili al link GitHub:\\ \url{https://github.com/gomax22/Computational-Docking}.


\noindent Di seguito è riportato il codice QR per l'accesso alla repository.

\vskip 1.5cm
\begin{figure}[H]
    \centering
    \includegraphics[scale=0.5]{images/frame.png}
\end{figure}

