\chapter*{Introduzione}
Nell'età moderna, le reti neurali rappresentano una realtà ormai affermata grazie al raggiungimento di obiettivi importanti e risultati significativi in svariati campi applicativi: dalla guida autonoma alle diagnosi mediche, passando per i sistemi di riconoscimento ottico dei caratteri. I motivi principali della crescita esponenziale dell'applicazione delle reti neurali sono dovuti alla capacità di estrazione delle caratteristiche in maniera autonoma dalla fase di apprendimento ed alla potenza di calcolo raggiunta dai calcolatori moderni, che permette di risolvere pesanti nuclei computazionali in tempi relativamente brevi, attraverso le schede grafiche di ultima generazione. 
Negli ultimi anni la ricerca di approcci avanzati di deep learning e lo sviluppo delle reti neurali convoluzionali si sono affermati anche nell'ambito del docking computazionale, provando a sostituirsi alle applicazioni esistenti basate su diversi approcci tradizionali, che uniscono le conoscenze fisiche, empiriche e stocastiche, derivanti dal dominio applicativo. 

Dopo aver introdotto alcune nozioni e concetti nell'ambito del docking molecolare, vengono discussi i passi, i requisiti essenziali e le applicazioni del docking computazionale, ponendo l'attenzione sulla correlazione tra l'esposizione ai pesticidi e la riduzione di colonie di api mellifera, oggetto del lavoro di Tirocinio a cui è fortemente ispirata l'applicazione proposta. 


A partire dall'introduzione dei concetti fondamentali del deep learning, l'elaborato si sofferma su un particolare software di docking computazionale che permette l'utilizzo delle reti neurali convoluzionali in diverse fasi del processo di docking (GNINA). Per poter mettere a fuoco le prestazioni di GNINA, viene introdotto un termine di paragone, rappresentato da AutoDock Vina, un software di docking basato su un approccio empirico, ampiamente utilizzato nelle analisi classiche di valutazione dell'interazione proteina-ligando. Vengono discussi i suddetti software ed i relativi approcci in maniera approfondita, entrando nei dettagli pertinenti della discussione ove necessario. 


Al fine di descrivere l'applicazione realizzata, dalla fase di sviluppo alla produzione dei risultati sperimentali, vengono motivate le tecnologie e le piattaforme utilizzate come le dipendenze interne od esterne all'applicazione.
L'elaborato propone una procedura di preparazione del dataset iniziale antecedente al processo di docking, per garantire una base uniforme ed equivalente ai software, ed una procedura di analisi e valutazione delle interazioni molecolari per mettere in evidenza le differenze tra i suddetti software e quindi i suddetti approcci, in termini di risultati ed interazioni osservate, al fine di comprendere meglio e sottolineare pregi e difetti dell'applicazione delle reti neurali convoluzionali. 

I risultati sperimentali evidenziano che le prestazioni della rete neurale convoluzionale applicata all'interno di GNINA, configurata in maniera predefinita, danno luogo a risultati sommariamente diversi rispetto a quelli forniti da un software tradizionale per analisi di blind docking. Ragion per cui, la soluzione ideale corrisponde ad un utilizzo complementare dei due approcci, sfruttando le caratteristiche positive di ciascuno.
